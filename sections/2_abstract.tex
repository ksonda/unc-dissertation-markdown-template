\abstract{ Continuous water supply (CWS) is an important operational aspiration for water service providers and governments in low and middle income countries for many reasons as part of the Sustainable Development Goal to ensure universal access to safely managed water services by 2030. It is a particularly important goal in India, where no large urban utility operates continuously, while central and state governments prioritize CWS in so-called '24/7' initiatives. One reason Three papers comprise this dissertation, each of which examines the operational implications of moving from intermittent to continuous water supply for households and water utilities, focusing specifically on the case of Amravati, in Maharashtra, India. The first paper examines the extent to which implementing CWS for 1/4th of the service area in Amravati impacted household water demand, and how this service level upgrade

The second paper 

The third paper examines how the continuous water su
 

The first essay examines the extent to which the increasing block tariff (IBT) implemented in Nairobi effectively targets subsidies to low-income households, one of the primary objectives of the IBT implemented by Nairobi City Water and Sewer Company (NCWSC) and the majority of utilities in Sub-Saharan Africa. Contrary to conventional wisdom, I find that high-income residential and nonresidential customers receive a disproportionate share of subsidies and that subsidy targeting is poor even among households with a private metered connection.


Subsidy incidence is, however, only one of several criteria that policy makers consider when designing tariffs. The second essay provides a systemic review of the literature on pricing water and sanitation services, identifying the ways in which the literature might inform tariff design and areas for future research. I find that the literature is diverse, fragmented, and focused primarily on industrialized countries. The majority of studies in the literature also examine two or fewer criteria, limiting the extent to which the literature characterizes the actual tradeoffs policy makers face when designing tariffs.
iv
The third essay develops a framework for simulating the performance of water and sanitation tariffs. I apply this framework to the case of Nairobi to examine the performance of five alternative tariff structures relative to the IBT implemented by NCWSC. I find that tariff alternatives with a uniform volumetric price perform equally well or better than IBT tariff alternatives at three levels of cost recovery. These findings add to a growing body of evidence that challenges commonly held perceptions about IBTs. These findings also underscore the benefits of getting utilities on path to full cost recovery, an essential component of financing the global aspiration to ensure universal access to high quality water and sanitation services.}	